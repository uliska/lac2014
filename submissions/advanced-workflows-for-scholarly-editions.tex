\documentclass[11pt,a4paper]{article}
\usepackage{lac2014}
\sloppy
\newenvironment{contentsmall}{\small}

% This package contains some stylings for the CFP in its submission form.
% For any further use (e.g. in proceedings) it can safely be commented.
\usepackage{ulCFPstyles}

\usepackage{hyperref}

\title{Version Control In Scholarly Editions}

%see lac2012.sty for how to format multiple authors!
\author
{Urs LISKA \and Janek WARCHOŁ
\\ openLilyLib.org / lilypondblog.org
\\ info@openlilylib.org
}



\begin{document}
\maketitle


\begin{abstract}
\begin{contentsmall}
A discussion of the role plain text based tools can play in the preparation of
scholarly editions. This paper is based on the authors' experience with an
edition prepared with LilyPond and \LaTeX, as well as the prospects with
another edition currently in preparation.
Focus is on collaborative workflows powered by version control -- a concept that
is to our knowledge nearly non-existent in current scholarly work,

\end{contentsmall}
\end{abstract}

\keywords{
\begin{contentsmall}
Musical engraving, scholarly editions, version control, work-flows.
\end{contentsmall}
}

\section{Introduction}
Recently we published a revised edition of the songs of Oskar Fried (1871\,--\,1941)%
\footnote{\url{http://en.wikipedia.org/wiki/Oskar\_Fried}}.
It was realized with a small team of only two musicologists (Urs Liska and Alexander
Gurdon) and one music engraver (Janek Warchoł).
It soon turned out that the project had a larger scale than expected, mainly for two
-- independent -- reasons: the revision dug up a lot more questions than we had thought of,
and he music's complexity was a real challenge for both the notation program and the 
engraver.

While it was clear right from the start that we'd use Lilypond%
\footnote{\url{http://www.lilypond.org}} as the notation program we only decided
halfway through the project to use \emph{versioning} to organize the plain text input
files LilyPond processes. This went along with the decision to use \LaTeX{} to compile
text and scores to a contigious volume.

As it turned out this was one of the most influential decisions we took in recent years
because preparing an edition with version control opens up completely new approaches to
collaborative work. While this style of working is obvious in software development it
applies \emph{equally} to scholarly editions and engraving scores.

\medskip
This paper is a rather personal report of our experiences with this project as well as
some comments on a project we're currently starting, and which is intended to become
a reference project for \emph{crowd engraving}, a more or less unknown concept that is
originally made possible by version control. We also wrote a number of blog posts on
\emph{Scores of Beauty}%
\footnote{\url{http://lilypondblog.org/category/fried-songs}}
which you may enjoy as additional reading.

\section{Fundamentals}
Editing plain text source files is a daunting task for people not used to it.
A majority of musicians, musicologists or music engravers are frightened when
confronted with a LilyPond input file, and often they have strong reservations
against the “nerdy” stuff. They want to \emph{see} what they're doing instead of entering
markup. However, plain text workflows offer substantial advantages over \textsc{wysiwyg}
approaches, and this is what we're going to talk about in this paper. Urs has written
an extensive essay on the subject%
\footnote{\url{http://lilypondblog.org/2013/07/plain-text-files-in-music/}}
so we can keep the general summaries quite short and concentrate on some specific stuff
about our collaborative experiences.

\subsection{Plain Text}

\subsection{Version Control}

\section{Collaboration}
% Hands-on experiences from the Fried songs edition

\subsection{Interaction Between Editor and Engraver}
% It is actually a hot topic in scholarly discussion
% to what extent the editor should also be an engraver.
% Scholarly editions usually let the editor only provide
% a model, while an engraver from the publishe prepares the
% score. By contrast commercial publishers usually require
% the editor to prepare a near-printable score document.

\subsection{Music entry}

\subsection{Proof-reading}

\subsection{Beautification}

\subsection{Maintainability After Publication}


\section{A New Project: Crowd Editing}

\subsection{Interaction and Peer Review}

\subsection{Using Programming to Simplify the Editors' Life}
% Mabe this subsection is too OT or project specific.
% This should be the first to be dropped if it's too long.


\section{Extending the Tools With Musicological Perspective}
% I'm not sure if this section will remain in the paper.
% It's slightly OT and might eat up too much space,
% OTOH I wouldn't want to miss it completely.
% Probably there should be a few notes here and there.
% Or maybe a short 'prospects' subsection in 'Conclusions'

\subsection{Graphical Editing for Frescobaldi}

\subsection{\texttt{\textbackslash annotate}}

\subsection{Embracing Music Encoding}


\section{Conclusions}

Concluding text.

\section{Acknowledgements}

Our thanks go to \ldots .

\end{document}
