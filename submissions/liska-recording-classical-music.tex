\documentclass[11pt,a4paper]{article}
\usepackage{lac2014}
\sloppy
\newenvironment{contentsmall}{\small}

\usepackage{fontspec}
\usepackage{polyglossia}
\usepackage{microtype}

\usepackage{xcolor}

\title{New Conceptions of Recording Classical Music}

%see lac2012.sty for how to format multiple authors!
\author
{Urs LISKA
\\ mail@ursliska.de
}



\begin{document}
\maketitle


\begin{abstract}
\begin{contentsmall}
Virtually all \textsc{daw} programs use “Arranger Windows” that are based on the metaphor of a multi-track magnetic tape recorder.

This paper presents a new conception of a work-flow which is much more suitable for recording classical music.
It outlines a set of tools that might one day turn into a software application.

The core concern is organizing and evaluating the recorded material, and specific interest lies in the use of version control for project management, offering perspectives for collaborative work-flows.
\end{contentsmall}
\end{abstract}

\keywords{
\begin{contentsmall}
Music production, classical music, work-flows.
\end{contentsmall}
}

\section{Introduction}

As a technically inclined musician I have always been interested in recording processes, from pubertal experiments on a four-track tape deck until today where music production on the computer is ubiquitous.
But in recent years I had the opportunity to produce (as a pianist) a large-scale series of professional recordings for Deutschlandradio Kultur, resulting in six CDs, recorded in 26 days over several years%
\footnote{Quotation or reference here?}.
The recording producer entrusted me with the complete recordings along with his notes so I could judge the first edits and very concisely express wishes and suggestions for improvements.
This way I had around 50 \textcolor{red}{CHECK} hours of recorded material to digest, which gave me more than enough reason to reflect on the workflows imposed by the usual \textsc{daw} tools.

This experience convinced me that the metaphor of a multi-track magnetic tape recorder is quite inappropriate for recording classical music.
In this paper I'm going to outline a new approach to that task which is based on the initial question: “What if the takes knew about their content instead only about their length?”

I'm going to make suggestions for a possible application design, although I currently don't (or rather can't) have any plans to implement any of this.
If this paper would motivate someone to start a project of its own or if it could raise discussion about integrating my ideas into existing software it would have achieved its goal.


\section{Section}

Text\footnote{Text of note.}, note at end of page.


\subsection{Subsection}


 
\subsubsection{Subsubsection}

Text of the subsubsection.
Text of the subsubsection.
Text of the subsubsection.
Text of the subsubsection.
Text of the subsubsection.
Text of the subsubsection.
Text of the subsubsection.
Text of the subsubsection (see Table~\ref{table1}).

Text of the subsubsection.
Text of the subsubsection.
Text of the subsubsection.
Text of the subsubsection.
Text of the subsubsection.
Text of the subsubsection.


\begin{table}[h]
 \begin{center}
\begin{tabular}{|l|l|}

      \hline
      Software & Features\\
      \hline\hline
      AA & Harddisk-Recording\\
      BB & MIDI Sequencing\\
      CC & Score Notation\\
      \hline

\end{tabular}
\caption{Example}\label{table1}
 \end{center}
\end{table}


Text of the subsubsection.
Text of the subsubsection.
Text of the subsubsection.
Text of the subsubsection.


\section{Section}

Text. Text. Text. Text. Text.
Text. Text. Text. Text. Text.

Text. Text. Text. Text. Text.
Text. Text. Text. Text. Text.
Text. Text. Text. Text. Text.
Text. Text. Text. Text. Text.
Text. Text. More text. Text. Text.

\section{Section}

Text. Text. Text. Text. Text.
Text. Text. Text. Text. Text.
Text. Text. Text. Text. Text.
Text. Text. Text. Text. Text.
Text. Text. More text. Text. Text.
Text. Text. Text. Text. Text.

Text. Text. Text. Text. Text.
Text. Text. Text. Text. Text.

\section{Conclusions}

Concluding text.

\section{Acknowledgements}

Our thanks go to \ldots .

\end{document}
