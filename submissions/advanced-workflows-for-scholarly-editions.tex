\documentclass[11pt,a4paper]{article}
\usepackage{lac2014}
\sloppy
\newenvironment{contentsmall}{\small}

\usepackage{fontspec}
\usepackage{polyglossia}
\usepackage{microtype}

\title{Plain Text Tools in Scholarly Musicology}

%see lac2012.sty for how to format multiple authors!
\author
{Urs LISKA \and Janek WARCHOŁ
\\ openLilyLib.org / lilypondblog.org
\\ info@openlilylib.org
}



\begin{document}
\maketitle


\begin{abstract}
\begin{contentsmall}
Plain text based tools and version control offer unique and fundamental advantages over
graphical approaches when preparing scholarly editions.
But it is difficult to convince musicologists, engravers and publishers getting their hands
dirty with source code.

This paper presents our experiences with two complex edition projects and outlines
how we are striving towards an ideal working environment by extending our favorite tools LilyPond
and LaTeX with specific solutions.

\end{contentsmall}
\end{abstract}

\keywords{
\begin{contentsmall}
Musical engraving, scholarly editions, version control, work-flows.
\end{contentsmall}
}

\section{Introduction}

\section{Fundamentals}

\subsection{Plain Text}

\subsection{Version Control}

\subsection{Collaboration}

\subsection{Maintainability Even After Publication}

\subsection{Integrating LaTeX and LilyPond}

\subsection{Summary: Potential for Musicological Work}

\section{A New Project: Crowd Editing}

\subsection{Using Programming to Simplify the Editors' Life}

\subsection{Summary: Potential for Musicological Work}

\section{Extending the Tools With Musicological Perspective}

\subsection{\texttt{\textbackslash annotate}}

\subsection{Graphical Editing for Frescobaldi}

\subsection{Embracing Music Encoding}


\section{Conclusions}

Concluding text.

\section{Acknowledgements}

Our thanks go to \ldots .

\end{document}
