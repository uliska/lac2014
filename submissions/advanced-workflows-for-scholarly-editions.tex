\documentclass[11pt,a4paper]{article}
\usepackage{lac2014}
\sloppy
\newenvironment{contentsmall}{\small}

% This package contains some stylings for the CFP in its submission form.
% For any further use (e.g. in proceedings) it can safely be commented.
\usepackage{ulCFPstyles}

\usepackage{hyperref}

\title{Version Control In Scholarly Editions}

%see lac2012.sty for how to format multiple authors!
\author
{Urs LISKA \and Janek WARCHOŁ
\\ openLilyLib.org / lilypondblog.org
\\ info@openlilylib.org
}



\begin{document}
\maketitle


\begin{abstract}
\begin{contentsmall}
A discussion of the role plain text based tools can play in the preparation of
scholarly editions. This paper is based on the authors' experience with an
edition prepared with LilyPond and \LaTeX, as well as the prospects with
another edition currently in preparation.
Focus is on collaborative workflows powered by version control -- a concept that
is to our knowledge nearly non-existent in current scholarly work,

\end{contentsmall}
\end{abstract}

\keywords{
\begin{contentsmall}
Musical engraving, scholarly editions, version control, workflows.
\end{contentsmall}
}

\section{Introduction}
Recently we published a revised edition of the songs of Oskar Fried (1871\,--\,1941)%
\footnote{\url{http://en.wikipedia.org/wiki/Oskar\_Fried}}.
It was realized with a small team of only two musicologists (Urs Liska and Alexander
Gurdon) and one music engraver (Janek Warchoł).
It soon turned out that the project had a larger scale than expected, mainly for two
-- independent -- reasons: the revision dug up a lot more questions than we had thought of,
and the music's complexity was a real challenge for both the notation program and the 
engraver.

While it was clear right from the start that we'd use Lilypond%
\footnote{\url{http://www.lilypond.org}} as the notation program we only decided
halfway through the project to use \emph{versioning} to organize the plain text input
files LilyPond processes. This went along with the decision to use \LaTeX{} to compile
text and scores to a contigious volume.

As it turned out this was one of the most influential decisions we took in recent years
because preparing an edition with version control opens up completely new approaches to
collaborative work. While this style of working is obvious in software development it
applies \emph{equally} to scholarly editions and engraving scores.

\medskip
This paper is a rather personal report of our experiences with this project as well as
some comments on a project we're currently starting, and which is intended to become
a reference project for \emph{crowd engraving}, a more or less unknown concept that is
originally made possible by version control. We also wrote a number of blog posts on
\emph{Scores of Beauty}%
\footnote{\url{http://lilypondblog.org/category/fried-songs}}
which you may enjoy as additional reading.

\section{Plain Text Fundamentals}
Editing plain text source files is a daunting task for people not used to it.
Many musicians, musicologists or music engravers are frightened when
confronted with a LilyPond input file, and often they have strong reservations
against the “nerdy” stuff. They want to \emph{see} what they're doing instead of entering
markup. However, plain text workflows offer substantial advantages over \textsc{wysiwyg}
approaches, and this is what we're going to talk about in this paper. Urs has written
an extensive essay on the subject%
\footnote{\url{http://lilypondblog.org/2013/07/plain-text-files-in-music/}}
so we can keep the general summary concise and concentrate on some specific stuff
about our collaborative experiences.

\subsection{Plain Text Files}
Plain text file formats are fundamentally different from binary file formats often
used by graphical \textsc{wysiwyg} programs. They have characteristics which we consider fundamentally superior. The key aspects are:

\paragraph{Transparency:}
Plain text files reveal every aspect of their content, everything is explicitly written
and readable. Anybody can inspect it and see what has been entered and if and how it is
manually tweaked. 

Binary files are opaque and conceal how they manage your input. They can only be
interpreted by their original programs or by programs that have specific input filters.

\paragraph{Separation of concerns:}
The explicit nature of plain text files provides a clean separation of content, meaning
and appearance. Through the use of explicit commands the content
of a document is clearly marked up semantically while this \emph{meaning} is completely
independent from its visual \emph{appearance}.

\paragraph{Readability and Stability:}
Plain text files are human readable, binary files are not. This simple statement has
considerable implications. Plain text files can be recovered to a much greater extent
than binary files after being damaged. And they are (at least conceptually) much better
equipped for evolution of applications and operating systems.

\paragraph{Editor independence:}
Plain text tools separate the tasks of editing, processing and displaying documents,
making it possible to choose different tools for different tasks. Plain text files
can principally be edited with \emph{any} text editor, which opens a wide array of
workflow options, for example editing on smartphones or through web interfaces.

\subsection{Version Control}
The possibility to maintain plain text files under \emph{version control} is on of
their most important aspects. Using version control for engraving and editing music
opens a full range of collaborative workflows.

\subsubsection{Basic Ideas}

\paragraph{Undo/Redo}
On a very basic level you may understand version control as an infinitely flexible
implementation of \emph{undo/redo} mechanisms. But different from the tools you will
know from most programs this isn't restricted to chronologically undoing steps one by
one. By contrast versioning allows you to inspect, revert or modify \emph{any} state
individually.

\paragraph{Line-by-line Comparison}
The foundation of version control is the ability to compare a project directory's (also
known as \emph{repository}) content \emph{line by line} and detect modifications on line
level. This is also the reason why versioning only works on plain text files.
Any modifications done during one editing session are packed as a \emph{changeset} and
can be accessed as a unit.

\paragraph{Project History}
The sum or chain of all changesets in a repository forms its \emph{history}. A user can 
at any time inspect any changeset, revert or modify it, or she can set the whole project
to that given state. This makes it possible to retrace the steps of a project.

\paragraph{Branching}
In a versioned project it is possible to work in \emph{branches}, which can be
understood as a kind of separate working sessions. Work done on one branch doesn't affect
work on other branches. So the main line of the project can always be kept in a consistent
state.

\medskip
But the most important potential of versioning is for collaborative workflows.


\section{Collaboration}
As mentioned earlier we made the step to version control halfway through our edition
project. Initially we maintained our project in a shared Dropbox folder, which already
enabled us to edit \emph{shared files} instead of having to exchange them via email --
which would have been plainly impossible with several hundred files. But this too
became nearly unmanageable because everybody had to scrupulously take care not to
overwrite a file that had been in the meantime been modified by someone else. We were
constantly making backups to prevent any serious data loss by such a mistake. But even
\emph{if} we were going to recover from a loss through our backups it would be complicated
to exactly restore everybody's differing work. This is where managing the project with Git%
\footnote{Git (\url{http://git-scm.com}) is one of many available version control systems.
As it is the one we used, any terms refer to Git's usage in case of doubt.}
made our lives \emph{significantly} easier!

\subsection{Distributed Work}
Git is a \textsl{distributed VCS} (version control system), and as such every collaborator
has a local copy of the complete repository%
\footnote{In centralised VCS such as CVS or Subversion there is only one central repository,
and any collaborator has to \emph{checkout} a given state of the project to work on.}.
This contains the full project history, and she can work locally without any restrictions.
Contacting the repository on the server is only necessary to synchronize by downloading
others' work, merging it with one's own and uploading the result.

It is even possible to have multiple remote repositories and to interact directly
between collaborators, avoiding the “central“ repository completely. This will be part
of a later section, for now we'll stick to the concept of one shared repository.

One of the main advantages of version control is the line-based comparison mentioned above.
This makes it possible to edit files \emph{simultaneously}, without any need for “locking”
files or similar workflow restrictions. As soon as parallel work is combined Git compares
the different states line by line and transparently merges everything that doesn't cause
“conflicts“. If such a conflict occurs, presumably because two editors have modified the
same line in a file, the system offers straightforward ways to resolve them manually.
This and the concept of the history allows multiple editors to work freely without the risk
of stepping on each other's toes. For example one editor could proof-read lyrics while
the engraver fine-tunes page layout.

\subsection{Editor and Engraver}
% It is actually a hot topic in scholarly discussion
% to what extent the editor should also be an engraver.
% Scholarly editions usually let the editor only provide
% a model, while an engraver from the publishe prepares the
% score. By contrast commercial publishers usually require
% the editor to prepare a near-printable score document.
Traditionally the tasks of the editor and the engraver were completely separated: the
editor somehow prepared an \emph{engraver's copy}, the engraver engaved a metal plate,
the editor did proof-reading, and the engraver finished the plates. In some contexts
this is still common practice, particularly with scholarly edition institutes, whose
editors usually don't do any engraving at all. On the other hand many commercial
publishers expect the editor to deliver (nearly) print-ready files that can be
finished in-house with minimal effort (and investment). However, all of these workflows
share the fact that files (or plates) have to be passed around, with strict
“locking strategies” in order to avoid data loss or conflicting situations. If an editor
notices a wrong note she can't simply fix it in the score file but has to send the
engraver an email asking him to apply the fix etc.

In a workflow based on plain text tools and versioning this is \emph{completely} different.
Everybody works on the same shared data repository and has full access to any file at any
time%
\footnote{We don't want to say that role-based privileges aren't useful in many cases
but want to stress the principal possibility here.}.
Each collaborator can apply any modification because 
\mbox{\emph{a)} this} won't compromise or lose data and
\mbox{\emph{b)} any} modification is meticulously documented in the changesets.
Particularly the latter aspect is important for the new workflow experience. If someone
else edits a file I can see precisely what he did and when (so it's possible to say
“Editor A fixed that pitch on Jan 23”, for example). And if \emph{I} edit files I can
do this without hesitation, because I know the others can immediately see the changes
and object to them if necessary.

One consequence of these possibilities is that the tasks and responsibilities of
(scholarly) editor and engraver can and will blur to some extent. If everybody has
permanent access to the complete data repository tasks don't have to be separated as
strictly as formerly. This makes a lot of sense because there usually are considerable
intersections of skillsets. Urs as the scholarly editor \emph{does} have strong sense
for typography, and Janek as the engraver \emph{does} have a sharp eye on details, so
he'll notice issues with the musical text too. In a versioned workflow both can do
as much of the other's task as both like and time allows, improving efficiency but also
the quality of the result. We think this option can have significant impact in today's
discussion of the relation between editor and engraver or editor, edition institute and
(commercial) publisher.

When preparing our edition of Oskar Fried's songs this really boosted our work. Being
able to inspect the other's work, to freely apply edits and not the least to encapsulate
work in branches made us work much more straightforwardly. While we had to be
extremely careful when still managing the Dropbox folder we could now just start working
on any given issue. Sharing ideas and reviewing each other's work proved very inspiring
and presumably raised the overall quality of our work. We wrote this several times, and
it may sound like a buzz-phrase, but it's literally true: We can't imagine how we could
have worked without these tools in our “earlier lives”.

Particularly the last phase of the project was very revealing. We had to do our final
proof-reading round after the scores were finished from the typographical perspective.
(Well, this was partly due to suboptimal planning on our part, but one will have this
situation nearly always to some extent.) Modifying a score that has already been tweaked
to perfection is a very delicate task because any change of the musical text can
severly mess up the layout. Fortunately LilyPond performed extremely well in this regard%
\footnote{\url{http://lilypondblog.org/2013/11/modifying-already-beautified-scores/}},
but version control made the process less daunting by magnitudes.
Urs applied fixes to the errors that he and his fellow musicologist found himself (instead
of having to ask Janek to do so) and wrapped each single edit in an individual \emph{commit}%
\footnote{“Commit” is Git's term for a changeset.},
using the commit message as a convenient place to document his opinion on the necessity of
typographic follow-up fixes. Janek could then inspect the chain of messages so he
immediately knew where to look particularly carefully. This ensured that no unintended
side-effect of edits would go unnoticed, making us quite relaxed as soon as we realized
that it really worked out. This situation even allowed Urs to apply or at least prepare
typographic fixes himself, knowing he didn't irreversibly mess anything up and that Janek
could easily identify and revert any change.
Maybe even more important was the fact that we could wrap all this work in separate branches.
The edition comprises 26 songs, and as all of them were handled in individual branches
we could freely work on whichever song was currently convenient, switching context at
will. Urs created a branch for any given song and added only the relevant commits to it.
When ready with that he handed the torch to Janek who would then check everything and
finally merge this branch to the main line again. This way we \emph{always} had the project
as a whole in a consistent state and were able to juggle our tasks around without and fear
and hesitation.

\subsection{Maintainability After Publication}
One more aspect to using plain text tools is also based on the meticulous documentation
of the project history. If the source code is written decently from the start it is
\emph{very} straightforward to maintain a project even after the initial publication.
As each fix or improvement will be stored in a changeset you can apply any changes
without messing up anything. There will always be a detailed record about which
changes have been added since publication, or between any given two prints. This is
particularly interesting with today's printing-on-demand strategies where new “runs”
may be “released” after only a few copies.


\subsection{Integrating Text and Music}
\textbf{TODO:} Should we have a chapter on LaTeX or not?\\
Actually it doesn't substantially add anything to the collaboration point,
but of course it adds to the general advantages for creating editions.

\subsection{Miscellaneous Tools}
In the context of collaborative workflows we also have to discuss further tools that
can be used to enhance the working experience.

\begin{itemize}
\item Github/Bitbucket\\
particularly pull requests and web interface
\item issue tracker\\
particularly when linked to commit messages
\item Trello\\
and other list based services
\item Relation between email/forum/trello/issue tracker communication\\
Describe our experience and make some suggestions on \emph{efficiently} making use
of the different approaches.
\end{itemize}

\section{Crowd Editing}
In the previous section we've talked about the positive impact that version control
has on the conception of collaborative work. While we were actually two people
working together this concept can easily be extended to more involved people.
Apart from improving traditional editing workflows this can be used to create a
completely new approach to editing scores which we'd give the buzz-phrase \emph{crowd
editing}. If any number of people can participate in a project under version control
this can easily be exploited to distribute the work among a “crowd” of contributors.
Different from the concept of having distinct roles that tend to blur through the
collaborative approach there can be numerous people doing essentially the same kind
of work in parallel, effectively “clustering” manpower. While we're convinced that
in most cases it won't make sense to rely completely on the swarm intelligence (so
you should still have a firm project leadership) and that this won't reduce the
absolute number of hours spent on a project, it can significantly reduce the overall
time to deliver by distributing these hours to be done in parallel.

\subsection{Music entry}
Music entry is a field that can straightforwardly be distributed among a large number
of contributors. You only need a few things to make this a reliable process: a common
understanding or a set of rules of coding style (e.\,g. indentation, amount of
comments, of barchecks etc.) and a usable communication infrastructure to keep track
of who is working on what. Appropriate tools for this are assignments that can be done in
issue trackers or on card like on a Trello board. This way the whole material can be
effectively divided through the number of available contributors.

\subsection{Proof-reading and Peer Review}
Version control encourages peer review workflows on all stages of a project's lifecycle.
It is not necessary to do peer review on more or less finished work, but it can be an
integral part of the process. A natural place to hook in is when merging working branches
into the main line. For example one can require the merge to be done by a different
collaborator than the one who did the work. For music entry it would for example be
appropriate to enter a certain amount of music and then ask someone else to review and
merge it. This ensures that only reviewed material goes into the project right from the
start. While this won't make a final proof-reading obsolete it will definitely improve
the overall quality of the result.

Online services like Github or Bitbucket provide \emph{pull requests} as a very convenient
tool for this concept. Whenever a branch has changed and differs from others one can
open a pull request, which will be a publicly visible announcement that would like changes
to be incorporated into the main line. The pull request can then be inspected, discussed,
modified and finally merged or rejected.

\subsection{Internal Programmability}
Plain text tools like LilyPond and \LaTeX{} offer a level of programmability that goes
beyond anything scripting tools can do for word processors or graphical notation programs.
This is because it is built in to the core of the tools. A simple
markup command highlighting text is created the same way as a complex
command including an image with a caption.

\section{Conclusions}\label{sec:conclusions}

Concluding text.


\subsection{Extending the Tools With Musicological Perspective}
% I'm not sure if this section will remain in the paper.
% It's slightly OT and might eat up too much space,
% OTOH I wouldn't want to miss it completely.
% Maybe only a short subsection here with \paragraphs instead
% of real sectioning commands

\subsubsection{Graphical Editing for Frescobaldi}

\subsubsection{\texttt{\textbackslash annotate}}

\subsubsection{Embracing Music Encoding}



\section{Acknowledgements}

Our thanks go to \ldots .

\end{document}
