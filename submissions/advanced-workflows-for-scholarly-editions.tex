\documentclass[11pt,a4paper]{article}
\usepackage{lac2014}
\sloppy
\newenvironment{contentsmall}{\small}

% This package contains some stylings for the CFP in its submission form.
% For any further use (e.g. in proceedings) it can safely be commented.
\usepackage{ulCFPstyles}

\usepackage{hyperref}

\title{Version Control In Scholarly Editions}

%see lac2012.sty for how to format multiple authors!
\author
{Urs LISKA \and Janek WARCHOŁ
\\ openLilyLib.org / lilypondblog.org
\\ info@openlilylib.org
}



\begin{document}
\maketitle


\begin{abstract}
\begin{contentsmall}
A discussion of the role plain text based tools can play in the preparation of
scholarly editions. This paper is based on the authors' experience with an
edition prepared with LilyPond and \LaTeX, as well as the prospects with
another edition currently in preparation.
Focus is on collaborative workflows powered by version control -- a concept that
is to our knowledge nearly non-existent in current scholarly work,

\end{contentsmall}
\end{abstract}

\keywords{
\begin{contentsmall}
Musical engraving, scholarly editions, version control, work-flows.
\end{contentsmall}
}

\section{Introduction}

\section{Fundamentals}
% The theoretical foundation.
% As short as possible.
% Link to the essay on Scores of Beauty

\subsection{Plain Text}

\subsection{Version Control}

\section{Collaboration}
% Hands-on experiences from the Fried songs edition

\subsection{Interaction Between Editor and Engraver}
% It is actually a hot topic in scholarly discussion
% to what extent the editor should also be an engraver.
% Scholarly editions usually let the editor only provide
% a model, while an engraver from the publishe prepares the
% score. By contrast commercial publishers usually require
% the editor to prepare a near-printable score document.

\subsection{Music entry}

\subsection{Proof-reading}

\subsection{Beautification}

\subsection{Maintainability After Publication}


\section{A New Project: Crowd Editing}

\subsection{Interaction and Peer Review}

\subsection{Using Programming to Simplify the Editors' Life}
% Mabe this subsection is too OT or project specific.
% This should be the first to be dropped if it's too long.


\section{Extending the Tools With Musicological Perspective}
% I'm not sure if this section will remain in the paper.
% It's slightly OT and might eat up too much space,
% OTOH I wouldn't want to miss it completely.
% Probably there should be a few notes here and there.
% Or maybe a short 'prospects' subsection in 'Conclusions'

\subsection{Graphical Editing for Frescobaldi}

\subsection{\texttt{\textbackslash annotate}}

\subsection{Embracing Music Encoding}


\section{Conclusions}

Concluding text.

\section{Acknowledgements}

Our thanks go to \ldots .

\end{document}
